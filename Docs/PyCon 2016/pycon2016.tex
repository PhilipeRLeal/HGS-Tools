\documentclass[hyperref={pdfpagelabels=false},compress,final]{beamer}
% \documentclass[hyperref={pdfpagelabels=false},compress,draft]{beamer}
%\let\Tiny=\tiny % get rid of some useless warnings...
\usepackage{lmodern} % also does the trick
%\usepackage{default}
\usepackage[english]{babel}
% some standard options
% \usepackage{verbatim}
\usepackage{graphicx} %If you want to include postscript graphics
\usepackage{graphics}
\usepackage{color}
\usepackage{epsfig}
\usepackage{minted}
\usepackage{url}
\usepackage{float}
%\usepackage{subfigure}
\usepackage{amsmath}
\usepackage{amsfonts}
\usepackage{amssymb}
\usepackage{longtable}
\usepackage{lipsum}
% \usepackage[latin1]{inputenc}
% \usepackage{times}
\usepackage{tikz}
\usetikzlibrary{arrows,shapes,fit}
%\usepackage{lineno}
%\usepackage[bf,nooneline,format=plain,indention=.3cm]{caption}
%\setlength{\captionmargin}{10pt}
%\usepackage[letterpaper,hmargin=1in,vmargin=1.25in]{geometry}
%\frenchspacing

% some tweaking for bibliography
\usepackage{natbib}
\renewcommand{\bibsection}{\subsubsection*{\bibname }}
\def\newblock{}

% local definitions
\newcommand{\foreign}[1]{{\it #1}}
\renewcommand{\emph}[1]{{\it #1}}

\newcommand{\myhw}{0.5\textwidth}
\newcommand{\myfigw}{0.9\textwidth}
\newcommand{\myw}{0.49\textwidth}
\newcommand{\lc}{life--cycle~}

% Equation references
\renewcommand{\eqref}[1]{Eq.~(\ref{#1})}
%\renewcommand{\vec}[1]{{\bf #1}}
% Abbreviations
\newcommand{\nn}{$N^2$}
\newcommand{\nnm}{$N^2_{max}$}
\newcommand{\zet}{$\zeta$}
\newcommand{\zetp}{$\zeta_{TP}$}
\newcommand{\dze}{$\Delta\zeta_{TP}$}
\newcommand{\htp}{$h_{TP}$}
\newcommand{\dttp}{$\Delta T_{TP}$}
\newcommand{\p}{$p$}
\renewcommand{\t}{$T$}
\renewcommand{\theta}{$\theta$}
% Directions
\newcommand{\degN}{\ensuremath{^\circ}N}
\newcommand{\degS}{\ensuremath{^\circ}S}
\newcommand{\degE}{\ensuremath{^\circ}E}
\newcommand{\degW}{\ensuremath{^\circ}W}
\newcommand{\degC}{\ensuremath{^\circ}C}
% other
\renewcommand{\deg}{\ensuremath{^\circ}} % use for degree in math mode
\newcommand{\st}{\ensuremath{^{st}}}
\newcommand{\nd}{\ensuremath{^{nd}}}
\newcommand{\rd}{\ensuremath{^{rd}}}
\renewcommand{\th}{\ensuremath{^{th}}}
\usepackage{verbatim} % the comment environment is in the verbatim package
% \renewenvironment{comment}{\begin{itemize}}{\end{itemize}}
\newcommand{\inlinecomment}[1]{{\noindent\sl#1}}
% \newcommand{\app}[1]{Appendix~\ref{#1}}
\newcommand{\app}[1]{Appendix~#1}
% Units (math environment)
\newcommand{\GB}{\ensuremath{\,\mbox{GB}}}
\newcommand{\ms}{\ensuremath{\,\mbox{m\;s}^{-1}}}
\newcommand{\mmday}{\ensuremath{\,\mbox{mm\;day}^{-1}}}
\newcommand{\kgmmday}{\ensuremath{\,\mbox{kg\;m}^{-2}\,\mbox{day}^{-1}}}
\newcommand{\kgmms}{\ensuremath{\,\mbox{kg\;m}^{-2}\,\mbox{s}^{-1}}}
\newcommand{\kgms}{\ensuremath{\,\mbox{kg\;m}^{-1}\,\mbox{s}^{-1}}}
\newcommand{\mss}{\ensuremath{\,\mbox{m\;s}^{-2}}}
\newcommand{\pvu}{\ensuremath{\,\mbox{PVU}}}
\newcommand{\kkm}{\ensuremath{\,\mbox{K\;km}^{-1}}}
\newcommand{\m}{\ensuremath{\,\mbox{m}}}
\newcommand{\s}{\ensuremath{\,\mbox{s}}}
\newcommand{\Kkm}{\ensuremath{\,\mbox{K\;km}^{-1}}}
\newcommand{\km}{\ensuremath{\,\mbox{km}}}
\newcommand{\K}{\ensuremath{\,\mbox{K}}}
\newcommand{\hPa}{\ensuremath{\,\mbox{hPa}}}
\newcommand{\mssq}{\ensuremath{\,\mbox{m\;s}^2}}
\newcommand{\JKmol}{\ensuremath{\,\mbox{J\;K}^{-1}\,\mbox{mol}^{-1}}}
\newcommand{\kgmol}{\ensuremath{\,\mbox{kg\;mol}^{-1}}}
\newcommand{\ssq}{\ensuremath{\times{}10^{-4}\,\mbox{s}^{-2}}}
% math-environment abbreviations
\newcommand{\ddz}[1]{\ensuremath{\frac{\partial #1}{\partial z}}}
\newcommand{\ddp}[1]{\ensuremath{\frac{\partial #1}{\partial p}}}

%
% New Environments
% standard bluish box
\newenvironment{myBox}[3][shadow=true]%
{\begin{center} \begin{minipage}{#2} \begin{beamerboxesrounded}[#1]{#3} \smallskip}%
      {\smallskip \end{beamerboxesrounded} \end{minipage} \end{center}}
% code box
\newenvironment{myCode}[3][shadow=true]%
{\begin{center} \begin{minipage}{#2} \begin{beamerboxesrounded}[#1]{#3}}%
      {\end{beamerboxesrounded} \end{minipage} \end{center}}
% N.B.: for some reason adding  \begin{minted}{python}/\end{minted} leads to errors...
% greenish box
\definecolor{lightgray}{rgb}{0.85,0.9,0.9}
\definecolor{darkgreen}{rgb}{0.0,0.4,0.2}
\setbeamercolor{upgreen}{fg=white,bg=darkgreen}
\setbeamercolor{lowgreen}{fg=black,bg=lightgray}
\newenvironment{myGreenBox}[3][upper=upgreen,lower=lowgreen,shadow=true]%
{\begin{center} \begin{minipage}{#2} \begin{beamerboxesrounded}[#1]{#3} \smallskip}%
      {\smallskip \end{beamerboxesrounded} \end{minipage} \end{center}}
% redish box
\definecolor{lightred}{rgb}{0.9,0.85,0.9}
\definecolor{darkred}{rgb}{0.5,0.0,0.2}
\setbeamercolor{upred}{fg=white,bg=darkred}
\setbeamercolor{lowred}{fg=black,bg=lightred}
\newenvironment{myRedBox}[3][upper=upred,lower=lowred,shadow=true]%
{\begin{center} \begin{minipage}{#2} \begin{beamerboxesrounded}[#1]{#3} \smallskip}%
      {\smallskip \end{beamerboxesrounded} \end{minipage} \end{center}}
%
% Hyphenation
\hyphenation{tro-po-pau-se}
\hyphenation{tro-po-pau-ses}
\hyphenation{pa-ra-me-ter-i-za-ti-ons}
\hyphenation{ra-di-o-son-de}
\hyphenation{me-ri-di-o-nal}
\hyphenation{a-na-ly-sis}
\hyphenation{ba-ro-cli-nic}
\hyphenation{geo-stro-phic}

%\DeclareMathOperator{\ln}{ln}
% increase line spacing by a factor of 1.5
%\renewcommand{\baselinestretch}{1.5}

\mode<presentation>{
\usetheme{Malmoe} % Malmoe Frankfurt
\usecolortheme{whale}
\usecolortheme{orchid}
\setbeamertemplate{bibliography item}[text]
\setbeamercovered{transparent}
}

\DeclareGraphicsExtensions{.png,.pdf,.jpg} %.pdf,.png,.jpg
\graphicspath{{figures/fullsize/}}
% \graphicspath{{figures/small/}}

\title[Large Ensembles and Batch Execution with Python]{A Class to manage Large Ensembles and \\ Batch Execution in Python}
\subtitle{\bigskip PyCon Canada}
\author[\href{http://www.physics.utoronto.ca/~aerler/}{Andre R. Erler} (\href{mailto:A.R.Erler@gmail.com}{A.R.Erler@gmail.com})]{Andre R. Erler}
\institute{\includegraphics[width=0.35\textwidth]{aquanty-Logo_transparent}\\ \bigskip}
\date{November $12^{th}$, 2016}
%\logo{\includegraphics[scale=0.15]{HD3}}

% \AtBeginSection[]
% {
%    \begin{frame}{Outline}
%        \tableofcontents[currentsection]
%    \end{frame}
% }

% *************************************************************************************************
\providecommand\thispdfpagelabel[1]{}

\begin{document}

% \section*{\ }

\begin{frame}
\titlepage
\end{frame}

\begin{frame}{Outline}
\tableofcontents%[pausesections]
\end{frame}


% ################################################################################################


\section{Introduction}
% \begin{frame}{Outline}
% \tableofcontents[currentsection]
% \end{frame}


\subsection[Repetitive Science]{Science is Repetitive}

\begin{frame}{Science is Repetitive}
 \begin{columns}
   \begin{column}{0.6\textwidth}
     \small \medskip\\
     To reach conclusive results, scientific experiments usually have to be repeated many times; either to establish statistical significance, or to test a range of parameter values.\\ 
     \bigskip 
     \hspace*{-1cm}
     \includegraphics[width=1.2\textwidth]{GC_SampleTray}
   \end{column}
   \begin{column}{0.4\textwidth}
     \vspace*{-1.cm}\\%\hspace*{-0.5cm}
     \includegraphics[width=1.2\textwidth]{geo_samples}\\
    \bigskip 
     \small Experiments are planned and conducted in large batches or so-called \textit{ensembles}.\\ \smallskip
     \begin{myGreenBox}[shadow=true]{0.85\textwidth}{Automation}
       \small It is therefore desirable to automate the most repetitive tasks, and to create tools for this purpose.
     \end{myGreenBox}
  \end{column}
 \end{columns}
\end{frame}


\subsection*{What I do}

\begin{frame}{\hspace{0.55\textwidth} Coupling Climate Models\\ \hspace*{0.55\textwidth} with Hydrologic Models}
  \begin{columns}
    \begin{column}{0.5\textwidth}
      \vspace*{-1.5cm}
      \includegraphics[width=\textwidth]{Ts_annual_ortho_ndl}\\
      {\scriptsize Surface Temperature in a Global and a nested Regional Climate Model}
      \begin{myBox}{0.9\textwidth}{}
        \footnotesize I run Climate and Hydrologic Models to the impact of climate change on water resources and generate projections of future hydro-climate.
      \end{myBox}
    \end{column}
    \begin{column}{0.5\textwidth}
      \vspace*{.25cm}\\
      \scriptsize Athabasca River watershed: \\ groundwater depth (top) and surface water depth (bottom)
      \includegraphics[width=\textwidth]{ARB.png}
    \end{column}
  \end{columns}
\end{frame}

\begin{frame}{High Performance Computing}
  \begin{columns}[T]
 \begin{column}{0.5\textwidth}
  \begin{itemize}
  \item High-resolution Climate simulations:\\ \smallskip
    \begin{itemize}
     \item 4 days on 128 cores and 300GB of storage per model year
     \item 36 ensemble members, 15 years each
    \end{itemize}
  \medskip %\pause
  \item Surface-Subsurface Hydrologic Simulations:\\ \smallskip
    \begin{itemize}
     \item 1 day on 2 cores per model year
     \item also 15 years each, 100+ ensemble members
    \end{itemize}
  \end{itemize}
  %\medskip
  %\begin{myBox}{0.8\textwidth}{}
    %\footnotesize Run on SciNet HPC Facility %All simulations were performed on the SciNet super-computing facility, requiring more than $10^6$ core hours and 40\,TB storage.
  %\end{myBox}
 \end{column}
 \begin{column}{0.5\textwidth}
   \vspace*{-.8cm}
  \includegraphics[width=1.1\textwidth]{TCS-1}
 \end{column}
\end{columns}
\end{frame}


\subsection*{Motivation}

\begin{frame}[fragile=singleslide]{\hspace*{0.5cm} Motivation: \\ \hspace*{0.5cm} Batch Processing}
  \begin{columns}
    \begin{column}{0.5\textwidth}
      \begin{itemize}
        \item<1-> In Computational Sciences repetitive tasks can be automated/scripted \smallskip
      \end{itemize}
    \end{column}
    \begin{column}{0.5\textwidth}
      \vspace*{-1.5cm}
        \onslide<2->{\begin{myRedBox}{0.9\textwidth}{\textbf{Boilerplate Code}}
          Python simplifies scripting a lot, but we still have a lot of boilerplate code! 
          This can be simplified further.
        \end{myRedBox}}
    \end{column}
  \end{columns}
        \small
      \begin{myCode}{\textwidth}{\textbf{Python is an Ideal Scripting Language}}
        \footnotesize
        \begin{minted}{python}
ensemble = [...] # a list of objects

# for loop iterating over list
tmp = [] # store results
for member in ensemble: # iterate over list
    tmp.append( result = member.operation(*args, **kwargs) ) 
ensemble = tmp

# list comprehension is already much shorter!
ensemble = [m.operation(*args, **kwargs) for m in ensemble]
        \end{minted}
      \end{myCode}
\end{frame}

\begin{frame}[fragile=singleslide]{\hspace*{0.5cm} Motivation: \\ \hspace*{0.5cm} Batch Processing}
  \begin{columns}
    \begin{column}{0.5\textwidth}
      \begin{itemize}
        \item<1-> In Computational Sciences repetitive tasks can be automated/scripted \smallskip
      \end{itemize}
    \end{column}
    \begin{column}{0.5\textwidth}
      \vspace*{-1.5cm}
        \onslide<1->{\begin{myRedBox}{0.9\textwidth}{\textbf{Boilerplate Code}}
          Python simplifies scripting a lot, but we still have a lot of boilerplate code! 
          This can be simplified further.
        \end{myRedBox}}
    \end{column}
  \end{columns}
        \small
      \begin{myCode}{\textwidth}{\textbf{Python is an Ideal Scripting Language}}
        \footnotesize
        \begin{minted}{python}
ensemble = [...] # a list of objects

# for loop iterating over list
tmp = [] # store results
for member in ensemble: # iterate over list
    tmp.append( result = member.operation(*args, **kwargs) ) 
ensemble = tmp

# list comprehension is already much shorter!
ensemble = [m.operation(*args, **kwargs) for m in ensemble]
        \end{minted}
      \end{myCode}
\end{frame}


% ################################################################################################


\section[\ Ensemble Class]{Batch Execution using an Ensemble Class}


\subsection*{Motivation}

\begin{frame}[fragile=singleslide]{\hspace*{0.5cm} Motivation: \\ \hspace*{0.5cm} Batch Processing}
  \begin{columns}
    \begin{column}{0.5\textwidth}
      \begin{itemize}
        \item<1-> In Computational Sciences repetitive tasks can be automated/scripted \smallskip
      \end{itemize}
    \end{column}
    \begin{column}{0.5\textwidth}
      \vspace*{-1.5cm}
        \onslide<1->{\begin{myGreenBox}{0.9\textwidth}{\textbf{The Ensemble Class}}
          %A container class on which member methods can be called can 
          %reduce boilerplate code and add more functionality
          \smallskip
          \begin{itemize}
            \item Redirect method calls to ensemble members \\ \smallskip
            \item Container Protocol
          \end{itemize}
        \end{myGreenBox}}
    \end{column}
  \end{columns}
        \small
      \begin{myCode}{\textwidth}{\textbf{And Ideal Use-case Example}}
        \footnotesize
        \begin{minted}{python}
ensemble = Ensemble(*[...]) # create Ensemble object

# apply member methods to entire ensemble
ensemble = ensemble.operation_1(*args, **kwargs)
...
ensemble = ensemble.operation_N(*args, **kwargs)

member_N = ensemble[n] # access elements by index
member_key = ensemble[key] # .. or by name/key
...
        \end{minted}
      \end{myCode}
\end{frame}


\subsection[Ensemble Class]{The Ensemble Class}

\begin{frame}[fragile=singleslide]{\hspace*{0.65\textwidth}The Ensemble Class}
  \begin{columns}
    \begin{column}{0.55\textwidth}
      \vspace*{-1.cm}
      \begin{myBox}{1.\textwidth}{Implementation Snippet}
        \small
        \begin{minted}{python}
class Ensemble(object):
  _members = None # members
  ...
  
  def __getitem__(self, i):
    # get individual members
    if isinstance(i, int):
      # access like list/tuple 
      return self._members[i]
    elif isinstance(i, string):
      ...
      
  def __iter__(self):
    # iterate over members
    mm = self._members
    return mm.__iter__()
    ...
    
        \end{minted}
      \end{myBox}
    \end{column}
    \begin{column}{0.45\textwidth}
      \medskip \\
      { \large Emulating the Python \\ \smallskip \textbf{\color{darkred} Container Type}:} \\ 
      \small
      \begin{enumerate}
        \item Support several built-in methods, such as \texttt{\_\_len\_\_}, \texttt{\_\_contains\_\_}, \texttt{\_\_iter\_\_} \medskip
        \item Item assignment like \texttt{list} or \texttt{dict} using \texttt{\_\_getitem\_\_} and \texttt{\_\_setitem\_\_}\medskip
        %\item Methods like \texttt{\_\_add\_\_} or \texttt{\_\_iadd\_\_} can also be implemented
      \end{enumerate}
      \begin{myBox}{0.85\textwidth}{Return Values}
        Calls to member methods return a new container or Ensemble with the results
       \end{myBox}
    \end{column}
  \end{columns}
\end{frame}


\begin{frame}[fragile=singleslide]{The Ensemble Class}
  \begin{columns}
    \begin{column}{0.45\textwidth}
      \medskip \\
      { \large Implementation of \\ \smallskip \textbf{\color{darkred} Member Methods}:} \\ 
      \small
      \begin{enumerate}
        \item Redirect calls to member methods/attributes by overloading \texttt{\_\_getattr\_\_} \medskip
        \item Execute call on all Ensemble members \medskip
        \item Return a new container or Ensemble with results
      \end{enumerate}
      \begin{myRedBox}{1.\textwidth}{Ensemble Wrapper}
        Methods require helper Class \texttt{EnsWrap} to apply arguments
       \end{myRedBox}
    \end{column}
    \begin{column}{0.55\textwidth}
      \vspace*{-1.cm}
      \begin{myBox}{1.\textwidth}{Implementation Snippet}
        \small
        \begin{minted}{python}
class Ensemble(object):
  _members = None # members
  ...
  
  def __getattr__(self, attr):
    # check if callable
    mem0 = self._members[0] 
    # assuming homogeneity...
    f = getattr(mem0,attr)
    if callable(f):
      # return Ensemble Wrapper
      v = EnsWrap(self,attr)
    else:
      # just return values
      v = [getattr(m,attr) \
        for m in self._members]
    return v
    
        \end{minted}
      \end{myBox}
    \end{column}
  \end{columns}
\end{frame}


\subsection*{Ensemble Wrapper}

\begin{frame}[fragile=singleslide]{\hspace*{0.65\textwidth}The Ensemble Class}
  \begin{columns}
    \begin{column}{0.55\textwidth}
      \vspace*{-1.cm}
      \begin{myBox}{1.\textwidth}{Implementation Snippet}
        \small
        \begin{minted}{python}
class EnsWrap(object):
  ...
  
  def __init__(self, ens, attr):
    _ensemble = ens # members
    _attr = attr # member method
      
  def __call__(self, **kwargs):
    # iterate over members
    new = Ensemble()
    for m in self._ensemble:
      f = getattr(m,self.attr)
      # execute member method
      new.append(f(**kwargs))      
    # return new ensemble
    return new
    ...
    
        \end{minted}
      \end{myBox}
    \end{column}
    \begin{column}{0.45\textwidth}
      \medskip \\
      { \large Implementation of the \\ \smallskip \textbf{\color{darkred} Ensemble Wrapper}:} \\ 
      \small
      \begin{enumerate}
        \item Initialize with ensemble members and called attribute \medskip
        \item Use \texttt{\_\_call\_\_} method to execute member method with arguments
      \end{enumerate}
      \begin{myGreenBox}{0.95\textwidth}{Parallelization}
        Simple parallelization using \texttt{multiprocessing.Pool}'s \texttt{apply\_async} can be applied
       \end{myGreenBox}
    \end{column}
  \end{columns}
\end{frame}



% ################################################################################################


\section{Argument Expansion}


\subsection*{Motivation}

\begin{frame}[fragile=singleslide]{How can we use Ensembles with Argument Lists}
  \begin{myCode}{\textwidth}{A Trivial Case}
    \footnotesize
    \begin{minted}{python}
# this defeats the purpose
members = [member.operation(arg1=arg) for arg in arg_list]
Ensemble(*members) # initialize new ensemble
    
# a better solution: pass list directly
ensemble.operation(arg1=arg_list, inner_list=['arg1'])
    \end{minted}
  \end{myCode}
  \bigskip
  
  \begin{columns}
    \begin{column}{0.4\textwidth}
    Argument lists can easily be implemented in the \texttt{\_\_call\_\_} method of the ensemble wrapper EnsWrap by creating a list of arguments for each member
    \end{column}
    \begin{column}{0.6\textwidth}
      \vspace*{-1.cm}
      \begin{myBox}{1.\textwidth}{}
        \small
        \begin{minted}{python}
# construct argument list
args_list = expandArgList(**kwargs)
# loop over lists
ens = self._ensemble
for m,args in zip(ens,args_list):
  f = getattr(m,self.attr)
  # execute member method with args
  new.append(f(**args))      
        \end{minted}
      \end{myBox}
    \end{column}
  \end{columns}
\end{frame}


\begin{frame}[fragile=singleslide]{How can we use Ensembles with Argument Lists}
  \begin{myCode}{\textwidth}{A More Complex Case: the Outer Product List}
    \footnotesize
    \begin{minted}{python}
# again, this defeats the purpose
arg_list = []
for arg1 in arg_list1: # construct arg_list from two lists
  for arg2 in arg_list2: # i.e. all possible combinations 
    arg_list.append(dict(arg1=arg1, arg2=arg2))
# apply list to ensemble
ensemble.operation(arg1=arg_list, inner_list=['arg1'])

# a better solution is to expand the lists internally
ensemble.operation(arg1=arg_list1, arg2=arg_list2, 
                    outer_list=['arg1','arg2'])
    \end{minted}
  \end{myCode}
  \myskip
  The \textbf{Outer Product} expansion of multiple argument lists creates argument lists with all possible combinations.
  \textbf{Outer Product} expansion, on the other hand, works like Python's \texttt{zip} function.
\end{frame}


\subsection*{Implementation}

\begin{frame}[fragile=singleslide]{\hspace*{0.03\textwidth}Outer Product\\ \hspace*{0.03\textwidth}Argument expansion}
  \begin{columns}
    \begin{column}{0.45\textwidth}
      \medskip \\
      { \large Recursive Implementation \\ \smallskip of {\usebeamercolor[fg]{frametitle} Outer Product}:} \\ 
      \small
      \begin{enumerate}
        \item Separate expansion arguments from others
        \item Recursively expand argument list
        \item Generate argument set for each ensemble member
      \end{enumerate}
    \end{column}
    \begin{column}{0.55\textwidth}
      \vspace*{-1.cm}
      \begin{myBox}{1.\textwidth}{Implementation Snippet}
        \small
        \begin{minted}{python}
class Ensemble(object):
  _members = None # members
  ...
  
  def __getattr__(self, attr):
    # check if callable
    mem0 = self._members[0] 
    # assuming homogeneity...
    f = getattr(mem0,attr)
    if callable(f):
      # return Ensemble Wrapper
      v = EnsWrap(self,attr)
    else:
      # just return values
      v = [getattr(m,attr) \
        for m in self._members]
    return v
    
        \end{minted}
      \end{myBox}
    \end{column}
  \end{columns}
\end{frame}


% ################################################################################################


\begin{frame}{Summary \& Conclusion}  
  \bigskip
  {\Large \color{teal} \textbf{The Ensemble Class} }\smallskip
  \begin{itemize}
    \item Functions like a \textbf{container} and exposes (parallelized) \textbf{member methods}
  \end{itemize}
   \medskip
  {\Large \usebeamercolor[fg]{frametitle} \textbf{Argument Expansion} }\smallskip
  \begin{itemize}
    \item Systematic expansion of argument lists from \textbf{inner} or \textbf{outer product}
  \end{itemize}
  \medskip
  \begin{myRedBox}{0.9\textwidth}{Sprint Project: Publish Ensemble Class}
    Create a stand-alone module with the Ensemble class and the argument expansion code for others to use, and add support for array-like item access/assignment
  \end{myRedBox}
\end{frame}

\begin{frame}
 \vfill
 \begin{center}
 \LARGE
 Thank You!\hspace{1.5cm}  $\sim$ \hspace{1.5cm} Questions?
 \end{center}
 \vfill
\end{frame}


%\appendix
%
%
%\subsection[Example]{The Ensemble in Operation}
%
%\begin{frame}[fragile=singleslide]{The Ensemble in Operation}
  %\begin{columns}
    %\begin{column}{0.5\textwidth}
      %Dataset/GCM specific parameters: \smallskip
      %\begin{itemize}
        %\item Input file types/names\smallskip
        %\item Interpolation tables/grid \smallskip
        %\item Variables / frequency \smallskip
      %\end{itemize}
        %\begin{myBox}{0.9\textwidth}{Multiple Datasets}
          %\begin{itemize}
            %\item \textit{\color{purple} Inheritance} for common procedures \smallskip
            %\item \textit{\color{teal} Polymorphism} for different procedures
          %\end{itemize}
        %\end{myBox}
    %\end{column}
    %\begin{column}{0.5\textwidth}
      %\vspace*{-1.25cm}
      %\begin{myBox}{1.0\textwidth}{}
                 \renewcommand{\theFancyVerbLine}{\sffamily\textcolor[rgb]{0.5,0.5,0.5}{\scriptsize\arabic{FancyVerbLine}}}
        %\footnotesize
        %\begin{minted}{python}
%class Dataset(object):
  %prefix = ''  # file prefix
  %vtable = 'Vtable'
  %gribname = 'GRIBFILE' # input
  %ungrib_exe = 'ungrib.exe'
  %ungrib_log = 'ungrib.exe.log'
  %...
  %def __init__(self, ...):
    %# type checking
    %...
  %def setup(self, src, ...):
    %...
  %def cleanup(self, tgt):
    %...
  %def extractDate(self, fname):
    %# match valid filenames
    %...
  %def ungrib(self, date, mytag):
    %# generate file for metgrid
    %...
        %\end{minted}
      %\end{myBox}
    %\end{column}
  %\end{columns}
%\end{frame}



\end{document}
